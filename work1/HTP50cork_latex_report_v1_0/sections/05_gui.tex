\chapter{GUI 구성 및 기능 설명}

\section{GUI 레이아웃 개요}
프로그램은 좌측 사이드바 입력 패널과 우측 3개 그래프 영역으로 구성된다(그림~\ref{fig:gui-layout}).

\begin{figure}[htbp]
  \centering
  \includegraphics[width=0.95\linewidth]{figures/fig_gui_layout.pdf}
  \caption{GUI 레이아웃 개념도}
  \label{fig:gui-layout}
\end{figure}

\section{사이드바 입력 패널}
\subsection{입력 항목}
사이드바는 크게 (1) Unit System, (2) Geometry, (3) Pyrolysis, (4) Material \& BC Data로 구성된다.
표~\ref{tab:sidebar}는 주요 입력과 기능을 정리한 것이다.

\begin{table}[htbp]
\centering
\caption{사이드바 주요 UI 요소 및 기능}
\label{tab:sidebar}
\begin{tabular}{p{35mm}p{105mm}}
\toprule
구성 요소 & 기능 \\
\midrule
Unit System & 단위계 선택(SI). 기본값 자동 적용 및 일부 파라미터 자동 갱신 \\
Geometry & $L_1, L_2, N_1, N_2, t_{final}, T_{init}$ 입력 \\
Pyrolysis: Density Change & ON: Advanced(밀도 감소 포함), OFF: Simple(Cp/k 동결만) \\
Pyrolysis: T\_critical & 열분해 동결 활성화 임계 온도($T_{crit}$) 설정 \\
Material \& BC Data & Cork/Metal 물성 및 $h(t), T_r(t)$ 데이터 CSV 로드/편집 \\
RUN SIMULATION & 입력 검증 후 해석 실행(열분해 포함/미포함 2회) \\
Progress/Status & 진행률 및 상태(Ready/Solving/Done/Error) 표시 \\
\bottomrule
\end{tabular}
\end{table}

\section{그래프 영역 및 버튼}
그래프는 3개 섹션으로 구성된다.
\begin{enumerate}
  \item Temperature History: 온도 이력(노드별)
  \item Specific Heat (Cp): 비열 변화(열분해 동결 영향 확인)
  \item Pyrolysis Comparison: 열분해 포함(실선) vs 미포함(점선) 비교
\end{enumerate}

각 그래프 섹션에는 공통적으로 \textbf{Options/CSV/PNG} 버튼이 제공된다.
\begin{itemize}
  \item Options: 표시 노드/색상/레이블 설정(팝업)
  \item CSV: 해당 그래프 데이터 내보내기
  \item PNG: 해당 그래프 이미지(고해상도) 내보내기
\end{itemize}
Temperature History에는 추가로 \textbf{Peak Temp} 버튼이 있으며, 선택 노드의 최고 온도와 도달 시간을 요약 표시한다.

\section{실행 흐름(스레딩/진행률)}
GUI 프리징을 방지하기 위해 계산은 워커 스레드에서 수행되며,
메인 스레드는 일정 주기(\SI{100}{ms})로 솔버의 현재 시간(\texttt{t\_current})을 폴링하여 진행률을 업데이트한다.
그림~\ref{fig:flow}는 실행 흐름을 개념적으로 나타낸다.

\begin{figure}[htbp]
  \centering
  \includegraphics[width=0.95\linewidth]{figures/fig_execution_flow.pdf}
  \caption{시뮬레이션 실행 흐름(개념도)}
  \label{fig:flow}
\end{figure}

\section{데이터 편집 팝업(Table Editor)}
재료/경계 데이터는 내장 Table Editor로 편집할 수 있으며, 주요 기능은 다음과 같다.
\begin{itemize}
  \item Add Row / Delete Row: 데이터 행 추가/삭제
  \item Plot Graph: 현재 데이터의 그래프 미리보기
  \item Save Default: 사용자 기본값(JSON)으로 영구 저장
  \item Save \& Close: 현재 세션에 적용 후 닫기
\end{itemize}

\section{그래프 옵션 팝업(Graph Options)}
Graph Options 창에서는 라인 단위로 (노드/색상/레이블)을 설정한다.
노드 지정은 숫자 외에도 \texttt{surf, mid/int, bot} 등의 별칭을 허용한다.
